\chapter*{Úvod}
\addcontentsline{toc}{chapter}{Úvod}  
\paragraph{}
Pravidelnou prácou systémového administrátora je inštalácia a príprava počítačov na prácu pre ľudí menej zbehlých v informačných technológiách. Toto zahŕňa všetko od umiestnenia a zapojenia počítača do elektrickej siete až po inštaláciu programov.

\paragraph{}
V tejto bakalárskej práci sa budeme zaoberať riešením problematiky inštalácie počítačového softvéru na väčšie množstvo staníc. Existuje viacero spoločností, ktoré vyvíjajú nástroje na riešenie tohto problému. Po dlhšom výskume sme zistili, že každé riešenie má svoje plusy a mínusy, a záleží na potrebách inštitúcie, ktoré riešenie zvolí. Avšak ani jedno z týchto riešení nie je vhodné pre našu prácu, keďže našim cieľom je vytvorenie jednoduchej aplikácie, ktorá má minimálne požiadavky na nastavenia. Z pomedzi známejších mien v tomto obore by sme spomenuli ako príklad \textbf{ZENworks 10 Configuration Management} od firmy Novell, ktoré na svoju prácu vyžaduje špeciálny server, čo pre mnoho menších inštitúcii ako sú základné a stredné školy je takmer nesplniteľná požiadavka. Práve tieto inštitúcie, ktoré nepotrebujú ťažkopádne riešenia sú cieľovou skupinou našej aplikácie. 

\paragraph{}
Cieľom tejto bakalárskej práce je analýza problému inštalácie, popísať možné riešenia tohto problému a implementovať jednoduchú aplikáciu, ktorá bude túto problematiku riešiť. Orientovať sa budeme na platformu Windows 7, keďže mnoho firiem a školských inštitúcii pracuje práve s týmto operačným systémom.
V prvej kapitole sa budeme venovať špecifikácii problému a návrhu aplikácie. V nasledujúcej kapitole objasníme použité technológie. V tretej kapitole, ktorá tvorí jadro práce, vysvetlíme jej implementáciu, popíšeme jednotlivé triedy a ukážeme kľúčové časti kódu. Poslednú kapitolu bude tvoriť dokumentácia k inštalácii a používaniu aplikácie.

\chapter{Špecifikácia problému a návrh riešenia}

\section{Zadefinovanie Problému}
V dnešnej dobe je často dôležitým nástrojom produktívneho zamestnanca vo firme počítač. Avšak aj napriek tomuto faktu mnoho z týchto zamestnancov sa do počítačov nevyzná a vyžadujú pomoc firemného administrátora pri inštalácii počítača alebo niektorých programov. Táto úloha je síce z pohľadu informatika triviálna, avšak zaberá obrovské množstvo vynaloženého času vzhľadom na jej nízku obtiažnosť. Administrátor musí stráviť niekoľko dní nad prípravou novokúpených počítačov, pred tým ako ich odovzdá zamestnancom.
Takisto v školách sa začínajú hromadne využívať počítače. Tu vzniká problém, že študenti sú zvedaví a ich zvedavosť často končí počítačom v nefunkčnom stave. Znova nainštalovanie počítaču z obrazu, ktorý obsahuje všetky programy by zabralo zbytočne veľa času, preto sa naša aplikácia sústredí na koncept inštalácie programov na vyžiadanie bez prítomnosti administrátora.
Problém, ktorému sa v tejto práci budeme venovať a ktorého výsledný program ho má čo najefektívnejšie riešiť, je ako automatizovať čo najviac práce s prípravou počítaču do funkčného stavu. Najčastejšie sa tento problém rieši pomocou klonovania disku, avšak toto riešenie je použiteľné len vtedy ak chceme mať rovnakú sadu programov na každej stanici. Takisto v takomto obraze disku sa ťažko robia dodatočné zmeny.

\section{Návrh riešenia}
Bohužiaľ inštalácie na operačnom systéme Windows nie sú homogénne. Dali by sa kategorizovať, bližšie popísane v sekcii \ref{sec:categories}, avšak výroba aplikácie, ktorá by vedela automaticky obslúžiť akúkoľvek inštaláciu, by bola takmer nemožná. Preto v rámci riešenia problému si budeme generovať vlastné inštalácie, ktoré budú mať spoločnú predlohu a vďaka tejto vlastnosti budeme môcť vyrobiť aplikáciu, ktorá tieto inštalácie bez zásahu používateľa nainštaluje.  Toto riešenie vyrobí balíčky, ktoré následne presunie na cieľovú stanicu, kde sa rozbalia a tým nainštalujú želané programy. Celý tento proces môže byť pre používateľa neviditeľný. Naša aplikácia bude takisto umožňovať inštaláciu balíka na vyžiadanie. Vďaka tejto funkcionalite základ systému, ktorý sa nainštaluje napríklad klonovaním, môže byť menší. Implementáciu a finálnu aplikáciu vieme rozdeliť na tieto časti:
\begin{itemize}
\item Odchytenie a uloženie priebehu inštalácie do balíčka
\item Inštalácia balíčka na cieľovej stanici
\end{itemize}
 
\subsection{Príprava balíčkov}
Ako prvé potrebujeme vytvoriť balíčky, ktoré naša aplikácia neskôr použije na inštalovanie programov. Aby sme vedeli inštalácie čo najpresnejšie zopakovať potrebujeme si pamätať súbory a registry, ktoré inštalácia vytvorila alebo upravila, zoznam programov, na ktorých je balík závislý a zoznam spustiteľných súborov, ktoré používa program obsiahnutý v našom balíku.

\paragraph{Porovnanie stavu pred a po inštalácií}
Tento prístup k problému pozostáva z vytvorenia zoznamu súborov na disku pred a po inštalácii a nasledovné vytvorenie rozdielu medzi týmito zoznamami. Avšak pri výrobe rozdielu by bolo treba kontrolovať aj čas poslednej úpravy pre súbory, ktoré neboli inštaláciou vytvorené ale len zmenené. V dnešnej dobe, kedy veľkosti pevných diskov sa pohybujú v stovkách gigabajtov, je tento proces zdĺhavý a neefektívny. 
\paragraph{Pomocou zabalenia inštalácie}
Ďalším možným riešením zisťovania zmien, ktoré inštalácia vykonala, je vytvorenie aplikácie, ktorá obalí inštalačný program a bude reagovať na jeho volania operačného systému na vytvorenie alebo zmenu súborov. Tento prístup je v ohľade na rýchlosť vykonávania nesmierne rýchly, avšak nie dokonalý. Pri implementácií by sme museli filtrovať volania operácií na dočasných súboroch, ktoré si inštalácia môže vytvárať počas svojho behu. Implementácia tohto prístupu by bola viazaná na konkrétny proces inštalácie a museli by sme ošetrovať špeciálne prípady, kedy inštalácia používa pomocné procesy, ktorých volania by nemuseli byť zachytené pôvodnou aplikáciou. 
\paragraph{Zachytením zmien na súborovom systéme}
Posledné riešenie, ktoré spomenieme a ktoré budeme implementovať, je pomocou štandardných knižníc .NET framework pre systémy Windows. V týchto knižniciach nájdeme mnoho nástrojov, ktoré nám pomôžu narábať so súborovým systémom. V tomto konkrétnom prípade použijeme triedu FileSystemWatcher, ktorá čaká na udalosti zo súborového systému a posiela ich ďalej do programu. Týmto spôsobom sa vyhneme vytváraniu kompletného zoznamu súborov na disku a zároveň zistíme zmeny v systéme vytvorené akýmkoľvek procesom v danom čase. S týmto prichádza aj mnoho negatív, ako zachytávanie udalostí od samotného operačného systému alebo programov bežiacich na pozadí nesúvisiacich s inštaláciou. Doteraz sme riešili ako zachytiť zmenu v súborovom systéme, avšak dôležitou súčasťou mnohých programov sú aj Windows registre. Spomínaný .NET framework obsahuje triedy na počúvanie udalostí v registroch, avšak tieto triedy neobsahujú dostatok informácií kde a aká zmena v registroch nastala aby sme ich mohli použiť. Preto pri zisťovaní zmien v registroch použijeme ich základne funkcie a to výpis podstromu  registrov do súboru a načítanie podstromu registrov zo súboru. Zmeny v registroch, ktoré vznikli počas inštalácie získame z rozdielu výpisov registrov pred a po inštalácií. Zistením zoznamu súborov a registrov, ktoré sa vyrobili alebo zmenili počas inštalácie dostaneme balíček, ktorý obsahuje všetky potrebné veci pre fungovanie daného programu.

\subsection{Inštalácia balíčkov}
Po tom ako pripravíme balíčky vyššie uvedeným spôsobom, treba spraviť inštaláciu programov z daných balíčkov. Balíčky majú nami zadefinovanú štruktúru, ktorá obsahuje nasledovné informácie:

\begin{itemize}
\item Súbory programu
\item Výpis registrov, ktoré program vyžaduje
\item Zoznam súborov programu a miesto ich inštalácie
\item Balíčky, ktoré treba mať nainštalované
\end{itemize}

Toto sú všetky informácie, ktoré sú potrebné na replikáciu inštalácie na iné stanice. Okrem týchto si v zozname všetkých balíčkov budeme pamätať zoznam spustiteľných súborov na ktoré sa ma vytvoriť odkaz a nastavenie typu inštalácie pri štarte systému alebo na vyžiadanie. Úlohou tejto časti aplikácie bude rozbaliť súbory z balíčka na miesto, ktoré bolo určené pri inštalácii, načítať hodnoty registrov podľa výpisu v balíčku a vytvoriť odkazy na spustiteľné súbory, ktoré by používateľa mohli zaujímať.

\subsection{Prenos balíčkov}
Ako posledné budeme v implementácií riešiť prenášanie balíčkov na cieľové stanice. Vychádzať budeme z predpokladu, že naša aplikácia bude používaná v prostredí obsahujúcom sieťové spojenie medzi stanicami a stanicu na ktorej sa budú skladovať všetky dostupné balíčky, pre potreby tejto práce ju budeme nazývať server. Užívateľská aplikácia vždy zo serveru stiahne zoznam balíčkov s odkazmi, ktoré má pripraviť pre používateľa, a neskôr pri inštalácií bude zo serveru sťahovať súbory inštalovaného balíku. Čo sa týka obnovovania zoznamu balíkov na serveri, bude administrátorskou povinnosťou presunúť súbor so zoznamom na server.

\subsection{Sériové kľúče}
Väčšina programov vyžaduje kúpenú licenciu, ktorá sa bežne kontroluje pomocou sériového kľúča. Tento kľúč sa zadáva pri inštalácií alebo prvom spustení. Ak sa zadáva pri prvom spustení, našu aplikáciu to neovplyvní, len administrátor bude zodpovedný za spustenie programu na každej stanici a zadanie kľúča. Ak sa kľúč zadáva pri inštalácií, naša aplikácia okopíruje program s jedným kľúčom na všetky stanice kam bude program nainštalovaný pomocou balíčka. Keďže v dnešnej dobe čím ďalej tým viac firiem poskytuje možnosť hromadného nákupu licencií, ktorým prislúcha len jeden sériový kľúč, rozhodli sme sa tento problém a jeho riešenie nechať na administrátorovi, ktorý bude musieť zariadiť aby boli dodržané všetky licenčné podmienky, ktoré sa týkajú kľúča.

\subsection{Užívateľské nastavenia}
Mnoho programov si v dnešnej dobe zapisuje nastavenia do používateľskej zložky umiestnenej na systémovom disku - C:\textbackslash Users\textbackslash\textless meno používateľa\textgreater\textbackslash. Táto zložka sa samozrejme mení podľa používateľa a preto treba aj pri inštalácií balíčka na toto myslieť a nainštalovať ho do používateľskej zložky práve prihláseného používateľa. Toto vyriešime jednoducho keď pri zapisovaní cesty skontrolujeme či neukazuje do používateľskej zložky a nahradíme ju našou značkou, ktorú neskôr nahradíme práve prihláseným používateľom.

\subsection{Kategorizovanie inštalácií}
\label{sec:categories}
Inštalácie v operačnom systéme Windows by sa dali rozdeliť do 3 kategórií:

\paragraph{Rozbalenie archívu}
Najjednoduchším typom inštalácií sú programy, ktoré sa šíria v archívoch typu .zip, .rar alebo .tar.gz. Pre nainštalovanie takéhoto programu, nám stačí rozbaliť archív na správne miesto pomocou jedného z mnohých špecializovaných programov.

\paragraph{Windows Installer}
Komponent určený na inštaláciu, údržbu a odstraňovanie softvéru na systéme Windows. Informácie o inštalácií a súbory programu sú zabalené v inštalačných balíkoch, ktoré sú vďaka ich koncovke známe ako MSI súbory. Pri tvorbe MSI balíku môže programátor pripraviť používateľské rozhranie v ktorom dopytuje informácie, ktoré sa neskôr použijú pri inštalácií ako sú dôležité cesty, časti balíku ktoré by používateľ chcel nainštalovať a podobne. Ďalšia vlastnosť týchto balíkov je odvolanie zmien v prípade chyby alebo zrušenia inštalácie na pokyn používateľa. Pre všetky štandardné operácie nad systémom má balík vygenerované opačné operácie, ktoré vrátia systém do stavu pred inštaláciou. Tieto balíky obsahujú mnohé ďalšie funkcie ako tichú inštaláciu, bez zobrazenia používateľského rozhrania, inštaláciu na vyžiadanie, balík vytvorí odkaz na program, avšak samotný program nainštaluje až keď sa používateľ rozhodne daný program spustiť. Takisto je možné počas inštalácie spúšťať vlastné skripty, napríklad na kontrolu sériového kľúča. 

\paragraph{Vlastná inštalácia}
Mnohé spoločnosti, ktoré vyvíjajú softvér, si implementujú vlastné inštalačné programy. Tieto programy sú navrhnuté špeciálne pre danú aplikáciu a často vyvíjané v programovacích jazykoch s natívnou podporou na rôznych operačných systémoch. Mávajú v sebe zakomponované algoritmy na kontrolu sériových kľúčov, overovacie požiadavky na vzdialené serveri ako ochranu proti pirátstvu a ďalšie funkcie prispôsobené požiadavkám spoločnosti.

%\section{Existujúce riešenia}
%Existuje viacero spoločností, ktoré vyvíjajú nástroje na manažovanie inštalácie a konfigurácie väčšieho počtu staníc. Po dlhšom výskume sme zistili, že každé má svoje plusy a mínusy, a záleží na potrebách inštitúcie, ktoré riešenie zvolí. Avšak ani jedno z týchto riešení nie je vhodné pre našu prácu, keďže našim cieľom je vytvorenie jednoduchej aplikácie, ktorá má minimálne požiadavky na nastavenia. Z pomedzi známejších mien v tomto obore by sme spomenuli ako príklad \textbf{ZENworks 10 Configuration Management} od firmy Novell, ktoré na svoju prácu vyžaduje špecifickú konfiguráciu serveru, čo pre mnoho menších inštitúcii ako sú základné a stredné je takmer nesplniteľná požiadavka. Práve tieto inštitúcie, ktoré nepotrebujú ťažkopádne riešenia sú cieľovou skupinou našej aplikácie. 
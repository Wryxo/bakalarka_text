\chapter{Technológie}

\section{C\#}
C\# sme zvolili kvôli podpore .NET frameworku. Ten nám poskytuje knižnice na narábanie s mnohými časťami operačného systému Windows. Programy napísané v tomto frameworku bežia v aplikačnom virtuálnom stroji, ktorý sa stará o bezpečnosť, správu pamäte a zachytávania výnimiek. Poskytuje nám triedy na prácu so súbormi, zložkami a registrami operačného systému. Vďaka ním dokážeme zachytiť a neskôr zopakovať inštalácie programov.

\section{HTTPS}
Zabezpečený hypertextový prenosový protokol je zabezpečená verzia komunikačného protokolu HTTP. Namiesto používania textovej komunikácie, komunikácia protokolu HTTPS je šifrovaná pomocou protokolu SSL alebo TLS. Táto komunikácia je závislá na digitálnych certifikátoch, bez ktorých sa stráca bezpečnosť protokolu HTTPS. V našej práci budeme tento protokol používať na prenos balíkov a konfiguračných súborov medzi používateľskými stanicami a serverom. Na šifrovanie použijeme protokol SSL.
\chapter{Implementácia}

\section{Zmeny súborového systému}
\paragraph{}
Základom našej aplikácie je .NET trieda FileSystemWatcher, ktorú na začiatku inicializujeme a nastavíme na počúvanie události na všetkých pevných diskoch počítaču. Táto trieda generuje události na ktoré reaguje zvyšok programu, existujú dve kategórie událostí na ktoré treba reagovať. Zmena súboru, ktorá zahrňuje vytvorenie, zmenu a zmazanie súboru a premenovanie súboru, ktoré sa uskotoční iba pri zmene názvu súboru. Táto trieda spĺňa takmer všetky naše požiadávky, avšak nevie rozpoznať, ktorý proces stojí za vytvorením daného súboru, čo spôsobuje problém, že môžu byť odchytené události zmien v súboroch od iných procesov ako je daná inštalácia, bežne totiž v operačnom systéme beží mnoho programov na pozadí, ktoré si vytvárajú a upravujú súbory.

\paragraph{Zmena súboru}
Pri tomto type událostí najprv zistíme, ktorá z troch vecí nastala. Pri vytvorení súboru si zapíšeme miesto kde bol tento súbor vytvorený do zoznamu súborov tohto balíku. Túto informáciu využijeme neskôr pri koprírovaní súborov do balíku a inštalácií balíku. Ak bola táto událosť zavolaná pri zmene súboru, skontrolujeme či už tento súbor v tomto balíku monitorujeme, ak nie tak si ho zapíšeme do zoznamu akoby bol novo vytvorený aby sme odzalohovali zmeny, ktoré v ňom nastali. Nakoniec ak táto událosť nastala pri zmazaní súboru, zistíme jeho výskyt v doterajšom zozname súborov a prípadne ho vymažeme aby sme sa neskôr vyhli chybe pri kopírovaní neexistujúceho súboru. Pri všetkých událostiach ignorujeme súbory, ktoré sme my vytvorili - zoznam súborov balíku, diferencie registrov.

\paragraph{Premenovanie súboru}
Pri tomto type události prejdeme náš doterajši zoznam nájdeme pôvodný názov a zmeníme ho na nový. Týmto sa opäť vyhneme problémom s neexistujúcimi súbormi.

\section{Rozdiel registrov}
\paragraph{}
K hodnotám registrov, ktoré program vytvoril alebo zmenil sa dostaneme pomocou vytvorenia rozdielu pôvodných registrov a registrov po nainštalovaní programu. K tomuto použijeme algoritmus na nájdenie najdlhšej spoločnej sekvencie, tento algoritmus nebudeme implementovať ale použijeme implementáciu **referencia**. Tento kus kódu dostane na vstup 2 textové súbory a vráti nám riadky v ktorých sa odlišujú. Ako vstup mu teda dodáme výpisy registrov v štandardnom formáte vytvorené pomocou program regedit. Ako výstup dostane výpis nových kľúčov a ich hodnôt vo formáte, ktorý program regedit dokáže prečítať a vytvoriť dané kľúče v registroch. Problém avšak vzniká ak inštalácia zmení hodnotu už vytvoreného kľúča, pretože algoritmus rozpozná zmenu len na riadku priamo s novom hodnotou a nie definíciou kľúča, ktorá bude vtedy identická s tou pred inštaláciou. Tento problém budeme musieť neskôr odstrániť, zatiaľ ho však odignorujeme, vzhľadom na to, že podľa prieskumu nie mnoho programov upravuje počas inštalácie predtým vytvorené registre.
\chapter{Dokumentácia}

\section{Inštalácia}
\subsection{Prerekvizity}
Aplikácia vyžaduje aby na každej podstanici bol nainštalovaný .NET framework verzia 4.0, v ktorom je aplikácia vyvíjaná. Táto verzia frameworku je automatický nainštalovaná v operačnom systéme Windows 7 alebo Windows 8. Ďalej aplikácia potrebuje aby stanica bola na spoločnej sieti so serverom, z ktorého bude sťahovať balíky. Tento server musí mať zložku s balíkmi prístupnu cez HTTP protokol a zároveň mať nainštalovaný SSL certifikát, ktorým sa bude identifikovať užívateľovi pri požiadavke na stiahnutie balíku.
\subsection{Aplikácia}
Zdrojovy kód aplikácie je dostupný na stránke https://github.com/Wryxo/bakalarka. Avšak je doporučené stiahnuť si inštalačný program zo stránky <TODO>, ktorý nainštaluje všetky potrebné súbory našej aplikácie a spustí prvotnú konfiguráciu. Takisto vytvorí odkaz po spustení, ktorý stiahne zo serveru obnovený zoznam balíkov.

\section{Konfigurácia}
Po skončení inštalácie sa spustí aplikácia, ktorá si vypýta cesty k zložkám potrebným pre správne fungovanie a adresu súborov na servery. Po konfigurácií sa aplikácia pokúsi stiahnuť zoznam balíčkov zo serveru, ktorý administrátor zadal.

\section{Spúšťanie}
V inštalašnej zložke programu sa nachádzajú 2 spustiteľné súbory:
\begin{itemize}
\item setitup-admin.exe - slúži pre administrátora na vytváranie balíkov. Vyžaduje administrátorské práva na spustenie, kvôli sledovaniu události na súborom systéme
\item setitup.exe - aplikácia, ktorá sa spúšťa keď užívateľ použije odkaz vytvorený našou aplikáciou. Aplikácia nainštaluje potrebný balík alebo spustí program vyžiadaný od užívateľa
\end{itemize}

\section{Správa balíkov}
\subsection{Vytvorenie balíku}
Po spustení administrátorskej aplikácie, sa zobrazí rozhranie v ktorom administrátor dokáže spustiť sledovanie inštalácie a kontrolovať v konzole priebeh sledovania a chybové správy. Po skončení sledovania bude prezentovaný formulárom na výber odkazov, ktoré majú byť dostupné užívateľovi, výberom balíkov na ktorých je práve vytváraný balík závislý a zvolením kedy má byť balík inštalovaný.

\subsection{Zoznam balíkov}
Zoznam balíkov je zapísaný v súbore na servery alebo v zložke s balíkmi. Administrátor môže manuálne meniť nastavenia balíkov, zmenou textových súborov v zložke s balíkmi. V týchto konfiguračných súbroch je zapísany zoznam balíkov, odkazov, ktoré sa majú vytvoriť a dokonca aj zoznam súborov určitého balíku.

\subsection{Inštalácia balíku}
Užívateľ môže nainštalovať balík spustením odkazu zo zložky s odkazmi. Tento odkaz zistí, či už je daný balík nainštalovaný a podľa toho sa rozhodne o ďalšom kroku. Ak balík ešte nebol nainštalovaný aplikácia stiahne potrebné súbory zo serveru a nakopíruje ich na potrebné miesto. V prípade, že balík bol nainštalovaný aplikácia spustí požadovaný program. Toto celé sa vykoná bez vedomia užívateľa.